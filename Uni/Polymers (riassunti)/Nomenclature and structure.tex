% !TEX encoding = UTF-8
% !TEX program = pdflatex
% !TEX spellcheck = it_IT

\documentclass[a4paper, 11pt]{article}

% sintassi
\usepackage[T1]{fontenc}
\usepackage[utf8]{inputenc}
\usepackage[english]{babel}
\DeclareUnicodeCharacter{00A0}{~}
\usepackage{fullpage}
%\usepackage[cochineal]{newtxmath}
%\usepackage{crimson,verbatim}
\usepackage{textcomp}
\usepackage{color,soul,hyperref,mwe}

% matematica e chimica

\usepackage{siunitx,amsmath,bm,chemfig}
\newcommand{\ud}{\mathop{}\!\mathrm{d}}
\sisetup{detect-all,math-rm = \ensuremath}
%\usepackage{arev}
\usepackage[charter]{mathdesign}


\setatomsep{2em}

{%
%\newcommand\setpolymerdelim[2]%{\def\delimleft{#1}\def\delimright{#2}} 
%\def\makebraces[#1,#2]#3#4#5{% 
%\edef\delimhalfdim{\the\dimexpr(#1+#2)/2}% 
%\edef\delimvshift{\the\dimexpr(#1-#2)/2}% 
%\chemmove{% 
%\node[at=(#4),yshift=(\delimvshift)] {$%\left\delimleft\vrule height\delimhalfdim %depth\delimhalfdim 
%width0pt\right.$};% 
%\node[at=(#5),yshift=(\delimvshift)] 
%{$\left.\vrule height\delimhalfdim depth\delimhalfdim 
%width0pt\right\delimright_{\rlap{$\scriptstyle#3$}}$};}} 
%\setpolymerdelim()



% tabelle e grafica
\usepackage{booktabs,graphicx,graphbox,subfig,caption,pdfpages,rotating,tabularx,lscape}
\captionsetup{font=small,
	format=hang,
	justification=centering,
	singlelinecheck=true,
	labelfont={sf,bf}	
}
\usepackage{float}
\floatstyle{plaintop}
\restylefloat{table}
\usepackage{multirow}
\usepackage{fancyhdr}
\pagestyle{fancy}
\fancyhead[LE,RO]{\textsl{\rightmark}}
\fancyhead[LO,RE]{\nouppercase{\leftmark}}
\fancyfoot[C]{\thepage}
\usepackage[margin=1in,headsep=.3in]{geometry}
\usepackage[suftesi]{frontespizio}
\usepackage{xparse}
\setlength\parindent{0pt}

\newenvironment{chapterabstract}{%
  \par\nobreak\noindent
  \textbf{\textit{Abstract}\hrulefill}\nobreak
  %\small
  \noindent\ignorespaces
}{%
  \par\nobreak\normalsize
  \vskip-\ht\strutbox\noindent
  \textbf{\hrulefill}%
}
\makeatletter
\NewDocumentCommand\headerspdf{ O {pages=-} m }{% [options for include pdf]{filename.pdf}
  \includepdf[%
    #1,
    pagecommand={\thispagestyle{fancy}},
    scale=1,
    ]{#2}}
\NewDocumentCommand\secpdf{somO{1}m}{% [short title]{section title}[page specification]{filename.pdf} --- possibly starred
  \clearpage
  \thispagestyle{fancy}%
  \includepdf[%
    pages=#4,
    pagecommand={%
      \IfBooleanTF{#1}{%
        \section*{#3}}{%
        \IfNoValueTF{#2}{%
          \section{#3}}{%
          \section[#2]{#3}}}},
    scale=.65,
    ]%
    {#5}}
\makeatother

\begin{document}

\begin{center}
\chemname
{\chemfig{-CH_2-CH(-[6]R)-}}
{\textbf{Vinyl class}} 
\end{center}
\qquad
\begin{center}
\chemname
{\chemfig{-CH_2-CH(-[6]H)-}}
{PE}
\qquad 
\chemname
{\chemfig{-CH_2-CH(-[6]CH_3)-}}
{PP}
\qquad
\chemname
{\chemfig{-CH_2-CH(-[6]*6(-=-=-=))-}}
{PS}
\qquad 
\chemname
{\chemfig{-CH_2-CH(-[6]Cl)-}}
{PVC}
\end{center}

\qquad

\begin{center}
\chemname
{\chemfig{-CH_2-C(-[2]H)(-[6]C(=[4]O)-O-R)-}}
{\textbf{Acrylics class}} \qquad 
\chemname
{\chemfig{-CH_2-C(-[2]CH_3)(-[6]C(=[4]O)-O-R)-}}
{\textbf{Methacrylics class}} \qquad 
\end{center}
\qquad 
\begin{center}
\chemname
{\chemfig{-CH_2-C(-[2]H)(-[6]C~N)-}}
{Polyacrylonitrile}
\qquad
\chemname
{\chemfig{-CH_2-C(-[2]CH_3)(-[6]C(=[4]O)-O-CH_3)-}}
{PMMA}
\end{center}

\qquad

\begin{center}
\chemname
{\chemfig{-CH_2-C(-[6]R)=CH-CH_2}}
{\textbf{Diene class}}
\end{center}
\qquad 
\begin{center}
\chemname
{\chemfig{-CH_2-C(-[6]H)=CH-CH_2}}
{{Polybutadiene}} \qquad 
\chemname
{\chemfig{-CH_2-C(-[6]CH_3)=CH-CH_2}}
{{Polyisoprene}} \qquad 
\chemname
{\chemfig{-CH_2-C(-[6]Cl)=CH-CH_2}}
{{Polychloroprene}} \qquad 
\end{center}

\newpage

\begin{center}
\chemname
{\chemfig{-CX_2-CR_2-}}
{\textbf{Vinylidenes class}}
\end{center}
\qquad 
\begin{center}
\chemname
{\chemfig{-CF_2-CF_2-}}
{{Polytetrafluoroethylene}} \qquad 
\end{center}

\newpage 
\begin{landscape}
\begin{table}[htp]
\centering
$
\begin{array}{llc}
\toprule
\textbf{Class} & \textbf{Selected polymer} & \textbf{Structure} \\
\midrule
\textbf{Polyoxides (ethers)} & \text{Polydimethyl phenylene oxide (PPO)} & \tiny{\chemfig{-O-*6(=(-[5]CH_3)-=(-)-=(-[3]CH_3)-)}}  \\
\text{} & \text{Polyoximethylene (POM)} & \tiny\chemfig{-O-CH_2-} \\
\midrule
\textbf{Polyesters} & \text{Polybutylene terephthalate (PBT)} & \tiny{\chemfig{-O-{(}CH_2{)_4}-[,1.5]O-C(=[2]O)-*6(-=-(-C(=[2]O)-)=-=)}} \\
\text{} & \text{Polyethylene terephthalate (PET)} & \tiny{\chemfig{-O-{(}CH_2{)_2}-[,1.5]O-C(=[2]O)-*6(-=-(-C(=[2]O)-)=-=)}} \\
\midrule
\textbf{Polycarbonates} & \text{Polyisopropylidene diphenylene carbonate (PC)} & \tiny{\chemfig{-O-*6(-=-(-C(-[2]CH_3)(-[6]CH_3)-*6(-=-(-O-C(=[2]O)-)=-=))=-=)}}\\
\midrule
\textbf{Polyamides} & \text{Polyhexamethylene adipamide (PA 66)} & \tiny\chemfig{-NH-{(}CH_2{)_6}-[,1.5]NH-C(=[2]O)-{(}CH_2{)_4}-[,1.5]C(=[2]O)-}\\
\midrule
\textbf{Polysulfones} & \text{Polyether sulfone} & \tiny\chemfig{-O-*6(-=-(-C(-[2]CH_3)(-[6]CH_3)-*6(-=-(-O-*6(-=-(-S(=[2]O)(=[6]O)-*6(-=-(-)=-=))=-=))=-=))=-=)}\\
 & \text{Polyphenylene sulfide} & \tiny\chemfig{-*6(-=-(-S-)=-=)} \\
\midrule
\textbf{Polyimides} & \text{Polyamide imide} & \tiny\chemfig{-*6(-=-(-NH-C(=[6]O)-[:30]*6(-=(-[:-30]C?(=[6]O))-(-[:30]C(=[2]O)-[:-30,2]N?-[8])=-=))=-=)} \\
\midrule
\textbf{Polyketones} & \text{Polyether etherketone (PEEK)} & \tiny\chemfig{-O-*6(-=-(-O-*6(-=-(-C(=[2]O)-*6(-=-(-)=-=))=-=))=-=-)} \\[3ex]
\bottomrule 
\end{array}
$
\end{table}
\end{landscape}

\newpage 

\begin{table}[htp]
\centering
$
\begin{array}{lcc}
\toprule
\textbf{Polymer} & \mathbf{T_m} \, (\text{°C}) & \mathbf{T_g} \, (\text{°C}) \\
\midrule 
\text{PVC} & - & 80 \\
\text{a-PS} & - & 100 \\
\text{PMMA} & - & 105 \\
\text{PPO} & - & 210 \\
\text{HDPE} & 140 & -110 \\
\text{LDPE} & 110 & -110 \\
\text{i-PP} & 165 & -10 \\
\text{POM} & 180 & -85 \\
\text{PBT} & 240 & 15 \\
\text{PA 66} & 265 & 20-70 \\
\text{PA 6} & 225 & 20-70 \\
\text{PET} & 265 & 70 \\
\text{PTFE} & 330 & -150 \\
\text{PC} & - & 150 \\
\text{polyphenylene sufide} & 285 & 85 \\
\text{PEEK} & 330 & 140 \\
\text{polyether sulfone} & - & 190 \\
\text{amide-imide} & - & 290 \\
\text{polyimide} & - & 400 \\
\midrule
\text{ABS (SAN/PAN)} & -70 & 110 \\
\text{COC} & - & 70-180 \\
\bottomrule 
\end{array}
$
\end{table}

\begin{equation}
\chi = 1-e^{-kt^n}
\end{equation}

\begin{equation}
	k = \frac{\nu^3 N \pi}{3}
\end{equation}

\begin{equation}
	M_n = \frac{\sum_x N_x M_x}{\sum_x N_x} = \frac{\sum_x w_x}{\sum_x w_x/M_x}
\end{equation}

\begin{equation}
	M_w = \frac{\sum_x N_x M_x^2}{\sum_x N_x M_x}=\frac{\sum_x w_x M_x}{\sum_x w_x}
\end{equation}

\begin{equation}
	M_z = \frac{\sum_x N_x M_x^3}{\sum_x N_x M_x^2}=\frac{\sum_x w_x M_x^2}{\sum_x w_x M_x}
\end{equation}

\begin{equation}
	M_v = \sqrt[\big{a}\,\,]{\frac{\sum_x N_x M_x^{1+a}}{\sum_x N_x M_x}}
\end{equation}

\begin{equation}
	T_g = T_g^\infty - \frac{K}{M_n}
\end{equation}

\begin{equation}
	\chi = \frac{\Delta H_m- \Delta H_c}{\Delta H_m^\text{ref}}\cdot 100
\end{equation}

\begin{equation}
	t = A\cdot e^\frac{E_\text{iso}}{2.303 \cdot RT} \quad \rightarrow \quad  \log{t} = A + \frac{E_\text{iso}}{2.303 \cdot RT}
\end{equation}

\begin{equation}
	\text{rate} = A\cdot e^\frac{-E_\text{dyn}}{2.303 \cdot RT} \quad \rightarrow \quad  \log{\text{rate}} = A -\frac{E_\text{iso}}{2.303 \cdot RT}
\end{equation}

\begin{equation}
	p_\text{gel,C} = \frac{2}{\bar{f}} 
\end{equation}

\begin{equation}
	p_\text{gel,FS} = \frac{1}{\bar{f}-1} 
\end{equation}

\begin{equation}
	\bar{f} = \frac{f_\text{iso}\cdot \text{MW}_\text{eq,iso}+ f_\text{alcohol}\cdot \text{MW}_\text{eq,alc}}{\text{MW}_\text{eq,iso} +\text{MW}_\text{eq,alc}}
\end{equation}

\chemfig{Cl-Si(-[2]CH_3)(-[6]Cl)-[,1.4]Cl + 3H_2 O \qquad \rightarrow \qquad  HO-Si(-[2]CH_3)(-[6]OH)-OH}  


\begin{equation}
	\log{a_T} = \frac{\Delta H}{R} \cdot \Bigl(\frac{1}{T}-\frac{1}{T_0}\Bigl)
\end{equation}

\end{document}