% !TEX encoding = UTF-8
% !TEX program = pdflatex
% !TEX spellcheck = it_IT

\documentclass[a4paper, 11pt]{article}

% sintassi
\usepackage[T1]{fontenc}
\usepackage[utf8]{inputenc}
\usepackage[italian]{babel}
\DeclareUnicodeCharacter{00A0}{~}
\usepackage{fullpage}
\usepackage[cochineal]{newtxmath}
\usepackage{crimson,verbatim}
\usepackage{textcomp}
\usepackage{color,soul,hyperref,mwe}

% matematica e chimica

\usepackage{siunitx,amsmath,bm,chemfig}
\newcommand{\ud}{\mathop{}\!\mathrm{d}}
\sisetup{detect-all,math-rm = \ensuremath}

\setatomsep{2em}

\newcommand\setpolymerdelim[2]{\def\delimleft{#1}\def\delimright{#2}} 
\def\makebraces[#1,#2]#3#4#5{% 
\edef\delimhalfdim{\the\dimexpr(#1+#2)/2}% 
\edef\delimvshift{\the\dimexpr(#1-#2)/2}% 
\chemmove{% 
\node[at=(#4),yshift=(\delimvshift)] {$\left\delimleft\vrule height\delimhalfdim depth\delimhalfdim 
width0pt\right.$};% 
\node[at=(#5),yshift=(\delimvshift)] 
{$\left.\vrule height\delimhalfdim depth\delimhalfdim 
width0pt\right\delimright_{\rlap{$\scriptstyle#3$}}$};}} 
\setpolymerdelim()

% tabelle e grafica
\usepackage{booktabs,graphicx,subfig,caption,pdfpages}
\captionsetup{font=small,
	format=hang,
	justification=centering,
	singlelinecheck=true,
	labelfont={sf,bf}	
}
\usepackage{float}
\floatstyle{plaintop}
\restylefloat{table}
\usepackage{multirow}
\usepackage{fancyhdr}
\pagestyle{fancy}
\fancyhead[LE,RO]{\textsl{\rightmark}}
\fancyhead[LO,RE]{\nouppercase{\leftmark}}
\fancyfoot[C]{\thepage}
\usepackage[margin=1in,headsep=.3in]{geometry}
\usepackage[suftesi]{frontespizio}
\usepackage{xparse}

\newenvironment{chapterabstract}{%
  \par\nobreak\noindent
  \textbf{\textit{Abstract}\hrulefill}\par\nobreak
  %\small
  \noindent\ignorespaces
}{%
  \par\nobreak\normalsize
  \vskip-\ht\strutbox\noindent
  \textbf{\hrulefill}%
}
\makeatletter
\NewDocumentCommand\headerspdf{ O {pages=-} m }{% [options for include pdf]{filename.pdf}
  \includepdf[%
    #1,
    pagecommand={\thispagestyle{fancy}},
    scale=1,
    ]{#2}}
\NewDocumentCommand\secpdf{somO{1}m}{% [short title]{section title}[page specification]{filename.pdf} --- possibly starred
  \clearpage
  \thispagestyle{fancy}%
  \includepdf[%
    pages=#4,
    pagecommand={%
      \IfBooleanTF{#1}{%
        \section*{#3}}{%
        \IfNoValueTF{#2}{%
          \section{#3}}{%
          \section[#2]{#3}}}},
    scale=.65,
    ]%
    {#5}}
\makeatother

\begin{document}

\section{Appendix}

\subsection{Density measuraments}

\begin{table}[htp]
\centering
$
\begin{array}{lcccc}
\toprule
\textbf{Sample} & \textbf{Specimen} & \textbf{Weight in air}\,(g) & \textbf{Weight in water}\,(g) & \mathbf{\rho}\,(kg/m^{3})\\
\midrule
\text{PET} & \text{a} & 0.7114 & 0.1684 & 1307\\
& \text{b} & 0.8203 & 0.1984 & 1316\\
& \text{c} & 0.8376 & 0.2029 & 1317\\
\midrule
\text{PET-T1} & \text{a} & 0.6929 & 0.1720 &  1327\\
& \text{b} & 0.8050 & 0.2015 & 1331\\
& \text{c} & 0.3537 & 0.0874 & 1325\\
& \text{d} & 0.5248 & 0.1301 & 1326\\
\midrule
\text{PET-T2} & \text{a} & 0.7491 & 0.1888 & 1334\\
& \text{b} & 0.7711 & 0.1959 & 1337\\
& \text{c} & 0.5891 & 0.1499 & 1338\\
\bottomrule
\end{array}
$
\caption{Weight and density measurements of PET bottles with different thermal treatment.}
\label{tab:admt}
\end{table}

\begin{table}[htp]
\centering
$
\begin{array}{lcccc}
\toprule
\textbf{Sample} & \textbf{Specimen} & \textbf{Weight in air}\,(g) & \textbf{Weight in water}\,(g) &  \mathbf{\rho}\,(kg/m^{3})\\
\midrule
\text{pure} & 1 & 1.1579 & 0.1817 & 1183\\
& 2 & 0.8851 & 0.1388 & 1183\\
\midrule
\text{P2} & \text{A} & 3.1649 & 0.6074 & 1235\\
& \text{D} & 4.6149 & 0.9135 & 1244\\
\midrule
\text{P2 after TT} & \text{C} & 2.1841 & 0.4442 & 1252\\
& \text{B} & 2.6498 & 0.5039 & 1232\\
\bottomrule
\end{array}
$
\caption{Weight and density of PMMA samples.}
\label{tab:apmma}
\end{table}

\end{document}