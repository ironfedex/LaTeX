% !TEX encoding = UTF-8
% !TEX program = pdflatex
% !TEX spellcheck = it_IT

\documentclass[a4paper, 11pt]{article}

% sintassi
\usepackage[T1]{fontenc}
\usepackage[utf8]{inputenc}
\usepackage[english]{babel}
\DeclareUnicodeCharacter{00A0}{~}
\usepackage{fullpage}
%\usepackage[cochineal]{newtxmath}
%\usepackage{crimson,verbatim}
\usepackage{textcomp}
\usepackage{color,soul,hyperref,mwe}

% matematica e chimica

\usepackage{siunitx,amsmath,bm,chemfig}
\newcommand{\ud}{\mathop{}\!\mathrm{d}}
\sisetup{detect-all,math-rm = \ensuremath}
%\usepackage{arev}
\usepackage[charter]{mathdesign}


\setatomsep{2em}

\newcommand\setpolymerdelim[2]{\def\delimleft{#1}\def\delimright{#2}} 
\def\makebraces[#1,#2]#3#4#5{% 
\edef\delimhalfdim{\the\dimexpr(#1+#2)/2}% 
\edef\delimvshift{\the\dimexpr(#1-#2)/2}% 
\chemmove{% 
\node[at=(#4),yshift=(\delimvshift)] {$\left\delimleft\vrule height\delimhalfdim depth\delimhalfdim 
width0pt\right.$};% 
\node[at=(#5),yshift=(\delimvshift)] 
{$\left.\vrule height\delimhalfdim depth\delimhalfdim 
width0pt\right\delimright_{\rlap{$\scriptstyle#3$}}$};}} 
\setpolymerdelim()

% tabelle e grafica
\usepackage{booktabs,graphicx,graphbox,subfig,caption,pdfpages}
\captionsetup{font=small,
	format=hang,
	justification=centering,
	singlelinecheck=true,
	labelfont={sf,bf}	
}
\usepackage{float}
\floatstyle{plaintop}
\restylefloat{table}
\usepackage{multirow}
\usepackage{fancyhdr}
\pagestyle{fancy}
\fancyhead[LE,RO]{\textsl{\rightmark}}
\fancyhead[LO,RE]{\nouppercase{\leftmark}}
\fancyfoot[C]{\thepage}
\usepackage[margin=1in,headsep=.3in]{geometry}
\usepackage[suftesi]{frontespizio}
\usepackage{xparse}
\setlength\parindent{0pt}

\newenvironment{chapterabstract}{%
  \par\nobreak\noindent
  \textbf{\textit{Abstract}\hrulefill}\nobreak
  %\small
  \noindent\ignorespaces
}{%
  \par\nobreak\normalsize
  \vskip-\ht\strutbox\noindent
  \textbf{\hrulefill}%
}
\makeatletter
\NewDocumentCommand\headerspdf{ O {pages=-} m }{% [options for include pdf]{filename.pdf}
  \includepdf[%
    #1,
    pagecommand={\thispagestyle{fancy}},
    scale=1,
    ]{#2}}
\NewDocumentCommand\secpdf{somO{1}m}{% [short title]{section title}[page specification]{filename.pdf} --- possibly starred
  \clearpage
  \thispagestyle{fancy}%
  \includepdf[%
    pages=#4,
    pagecommand={%
      \IfBooleanTF{#1}{%
        \section*{#3}}{%
        \IfNoValueTF{#2}{%
          \section{#3}}{%
          \section[#2]{#3}}}},
    scale=.65,
    ]%
    {#5}}
\makeatother

\begin{document}

\includepdf[pages=-]{frontespizio4.pdf}

\begin{chapterabstract}

Lorem ipsum dolor sit amet, consectetur adipiscing elit. Vivamus at est non arcu dapibus eleifend. In facilisis, erat ullamcorper condimentum venenatis, magna mi consequat enim, ac faucibus velit libero ut sapien. Nam fringilla elit at magna imperdiet dapibus. Vestibulum malesuada fringilla placerat. Suspendisse potenti. Nam ac vehicula nibh, eget suscipit purus. Curabitur tempus erat id pulvinar consectetur. Etiam quis diam magna. Etiam feugiat vulputate consequat. Nulla libero ligula, tempor quis mollis in, placerat vitae nisi. Proin ornare ut eros ac rhoncus. Integer et metus non mi pharetra sagittis. Praesent dignissim erat vel magna efficitur, sed efficitur mi viverra. Interdum et malesuada fames ac ante ipsum primis in faucibus. Vivamus non ipsum non purus feugiat interdum non vel odio.

\end{chapterabstract} 



\section{Introduction}

\section{Materials and methods}
\subsection{Sample preparation for compression molding and injection compounding}

Two types of Polypropylene (PP) have been poured together in order to obtain a mixture with higher properties with respect to the matrix. For our purpose 15\% in weight of flame retardant PP pellet has been added to isotactic PP (PPH-B-10-FB). In Figure \ref{fig:datasheet} is reported the technical datasheet of PPH-B-10-FB.
For compression molding a stainless steel plates with a frame of $12\times 12\, \text{mm}^2$ has been filled with 34 g of the mixture composed by 27.22 g of isotactic PP and 6.68 g of flame retardant PP. 
For injection compounding 40 g of the mixture have been weighed. In this case the misture was composed by 34.01 g of isotactic PP and 5.99 g of flame retardant PP.
\begin{figure}[h!]
	\centering
	{\includegraphics[scale=0.6]{datasheetPP}}
	\captionsetup{justification=centering}
	\caption{Technical datasheet of PPH-B-10-FB}
	\label{fig:datasheet}
\end{figure}\\

\subsection {Injection molding: sample evaluation}

Two groups of specimens have been analyzed to identify specific materials: ISO dumbbell specimens and ASTM dumbbell specimes. For each group several samples have been provided in order to recognize a specific polymer. In Table \ref{tab:materials} is reported the list of polymers recognized. 

\begin{table}[htp]
	\centering
	$
	\begin{array}{ll}
	\toprule
	\textbf{ASTM dumbbell} & \textbf{ISO dumbbell}  \\
	\midrule
	\text{ABS} & \text{POM bi-injected}\\
        \text{COC} & \text{PA11 mono-injected}\\
	\text{PP} & \text{PP-GF 30 (white) and PP-GF35 (black)}\\
	\text{HDPE Eltex yellow} & \text{PA6-GF50}\\
	\text{PE/PP blend} & \text{}\\
	\text{PA11} & \text{}\\
	\text{TPU} & \text{}\\
	\bottomrule
	\end{array}
	$
	\caption{Materials used in sample evaluation}
	\label{tab:materials}
\end{table}



\end{document}
